% !TeX program = xelatex
\documentclass{article}
\usepackage{xeCJK}
\usepackage{tikz}
\usepackage[outline]{contour}
\setCJKmainfont{Kaiti SC}
% \setCJKmainfont{STKaiti}

\newcommand\Grid[1]{%
 \tikz[baseline=(char.base)]{%
  \draw[xstep=1ex,ystep=1ex,help lines] (-1ex,-1ex) grid (1ex,1ex);
  \node[inner sep=0pt] (char) at (0,0) {#1};
 }%
}
\pagenumbering{gobble}
\begin{document}
\begin{center}
    \Huge 题西林壁 \\
\end{center}
\begin{flushright}
    \huge 宋 $\bullet$ 苏轼 \\
\end{flushright}
\begin{center}
    \Huge 横看成岭侧成峰 \\
    \Huge 远近高低各不同 \\
    \Huge 不识庐山真面目 \\
    \Huge 只缘身在此山中 \\
\end{center}
\noindent\makebox[\linewidth]{\rule{\paperwidth}{0.4pt}}
\begin{center}
    \Huge 题西林壁 \\
\end{center}
\begin{flushright}
    \huge 宋 $\bullet$ 苏轼 \\
\end{flushright}
\begin{center}
    \Huge \Grid{横}\Grid{看}\Grid{成}\Grid{岭}\Grid{侧}\Grid{成}\Grid{峰} \\
    \Huge \Grid{远}\Grid{近}\Grid{高}\Grid{低}\Grid{各}\Grid{不}\Grid{同} \\
    \Huge \Grid{不}\Grid{识}\Grid{庐}\Grid{山}\Grid{真}\Grid{面}\Grid{目} \\
    \Huge \Grid{只}\Grid{缘}\Grid{身}\Grid{在}\Grid{此}\Grid{山}\Grid{中} \\
\end{center}
\noindent\makebox[\linewidth]{\rule{\paperwidth}{0.4pt}}
人们的思维、认知、意识、观念天然地局限于已知范围内,这是主观精神世界的安全区,使之免于未知的恐惧、信仰的冲击,但是这也会导致一个人很难发现自身在思维、认知、意识、观念上的问题,即使这些问题影响了个人和群体的行为、习惯、性格、命运。
\end{document}